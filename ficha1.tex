\documentclass{beamer}
%
% Choose how your presentation looks.
%
% For more themes, color themes and font themes, see:
% http://deic.uab.es/~iblanes/beamer_gallery/index_by_theme.html
%
\mode<presentation>
{
  \usetheme{default}      % or try Darmstadt, Madrid, Warsaw, ...
  \usecolortheme{default} % or try albatross, beaver, crane, ...
  \usefonttheme{default}  % or try serif, structurebold, ...
  \setbeamertemplate{navigation symbols}{}
  \setbeamertemplate{caption}[numbered]
} 

\usepackage[brazil]{babel}
\usepackage[utf8]{inputenc}
\usepackage[T1]{fontenc}

\usepackage{tikz}
\usepackage{siunitx}

\title[]{Matriz Simétrica}
\author{Tópicos em controle}
\institute{IME USP}
\date{22 de agosto de 2020}

\begin{document}



\begin{frame}
  \titlepage
\end{frame}

\begin{frame}{Um resultado sobre matriz simétrica}

  Se $A$ é uma matriz simétrica então  
  $$ \text{Ker}A = \text{Im}A^{\bot}$$
  \begin{itemize}
    \item $\text{Ker}A = \{x\in \mathbb{R}^n \text{ : }  Ax=0\}$
    \item $\text{Im}A^\bot =\{ x \in \mathbb{R}^n \text{ : } x^T y = 0 \forall y \in \text{Im}A\}$
  \end{itemize}
  
\end{frame}

\begin{frame}

Se $x\in \text{Ker}A$ então:
\begin{gather*}
  \forall y = Av \implies x^Ty = x^T A v = (A^Tx)^Tv \\
  = (Ax)^Tv = 0 \implies x\in \text{Im}A^\bot \\
  \text{Ker}A \subseteq \text{Im}A^{\bot}
\end{gather*}
Se $x\in \text{Im}A^\bot$
\begin{gather*}
  0=x^TAe_i =(A^Tx)^T.e_i = (Ax)^T.e_i=0 \implies \\
  Ax=0 \implies \text{Im}A^\bot \subseteq \text{Ker}A
\end{gather*}


\end{frame}


\end{document}
