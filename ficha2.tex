\documentclass{beamer}
%
% Choose how your presentation looks.
%
% For more themes, color themes and font themes, see:
% http://deic.uab.es/~iblanes/beamer_gallery/index_by_theme.html
%
\mode<presentation>
{
  \usetheme{default}      % or try Darmstadt, Madrid, Warsaw, ...
  \usecolortheme{default} % or try albatross, beaver, crane, ...
  \usefonttheme{default}  % or try serif, structurebold, ...
  \setbeamertemplate{navigation symbols}{}
  \setbeamertemplate{caption}[numbered]
} 

\usepackage[brazil]{babel}
\usepackage[utf8]{inputenc}
\usepackage[T1]{fontenc}

\usepackage{tikz}
\usepackage{siunitx}

\title[]{Lema 1}
\author{Tópicos em controle}
\institute{IME USP}
\date{24 de agosto de 2020}

\begin{document}



\begin{frame}
  \titlepage
\end{frame}

\begin{frame}{Lema 1}

 O lema 1 é muito interessante pois associa propriedades de um operador linear 
 num espaço de dimensão infinita às propriedades de uma matriz!

 Se $G: \mathbb{R} \to \mathbb{R}^{n\times m}$ é contínua e $\mathcal{C}([t_0,t_1], \mathbb{R}^m)$
 é o nosso espaço vetorial de funções, então operador:
 $$ \mathcal{L}(u(\cdot) )=\int_{t_0}^{t_1}G(s)u(s)ds $$
 é um operador linear de 
 $\mathcal{C}([t_0,t_1], \mathbb{R}^m)$ em $\mathbb{R}^n$
  
\end{frame}
\begin{frame}
\textbf{Lema 1:} Se $W(t_0,t_1) = \int_{t_0}^{t_1}G(t)G^T(t)dt$, então 
$\text{Im}\mathcal{L} = \text{Im}W(t_0,t_1)$.

\textit{Prova: }
1. $\text{Im}W(t_0,t_1) \subseteq \text{Im}\mathcal{L}$, se $\mathbf{x} \in 
\text{Im}W(t_0,t_1)$ então temos um $\mathbf{y} \in \mathbb{R}^n$ com
$W(t_0,t_1)\mathbf{y} = \mathbf{x}$ e da definição de $W$ temos
\begin{gather*}
  W(t_0,t_1)\mathbf{y} = \int_{t_0}^{t_1}G(t)G^T(t)\mathbf{y}dt = \mathbf{x}\\
  = \mathcal{L}(u(\cdot)) \text{ para } u(t) = G^T(t)\mathbf{y}
\end{gather*}
\end{frame}

\begin{frame}
  2. $\text{Im}\mathcal{L} \subseteq \text{Im}W(t_0,t_1)$

  Se $\mathbf{x} \not\in \text{Im}W(t_0,t_1)$, usando o fato de que $W$ é simétrica,
  podemos encontrar um $\mathbf{y}\neq 0$ tal que $W(t_0,t_1)\mathbf{y}=0$ e 
  $\mathbf{y}^T\mathbf{x} \neq 0$.

  Notemos então que:
  \begin{gather*}
    0= \mathbf{y}^TW(t_0,t_1)\mathbf{y} = \int_{t_0}^{t_1}(G^T(t)\mathbf{y})^T(G^T(t)\mathbf{y})dt \\
    = \int_{t_0}^{t_1}\|G(t)^T\mathbf{y}\|^2 \implies \\
    G(t)^T\mathbf{y}=0 \forall t\in[t_0,t_1]
  \end{gather*}
\end{frame}

\begin{frame}
  Se $\mathbf{x}$ estivesse em $\text{Im}\mathcal{L}$ então:
  \begin{gather*}
    \mathbf{x} = \int_{t_0}^{t_1}G(t)u(t)dt \implies \\
    \mathbf{y}^T\mathbf{x}=\int_{t_0}^{t_1}\mathbf{y}^TG(t)u(t)dt = \\
    = \int_{t_0}^{t_1} [G^T(t)\mathbf{y}]^Tu(t) dt = 0
  \end{gather*}
  Que contraria como foi escolhido o $\mathbf{y}$
  assim fica demosntrado o lema $\qed$
\end{frame}

\begin{frame}
  Vamos considerar uma aplicação deste Lema.
  considere a equação:
  $$ \dot{\mathbf{z}}(t) = B(t)u(t)$$
  então 
  $$ \mathbf{z}(t) = \mathbf{z}_0 + \int_{t_0}^t B(s)u(s)ds \implies \mathbf{z}(t)-\mathbf{z}_0 = \int_{t_0}^t B(s)u(s)ds$$
De forma que pelo Lema 1 existe $u(t)$ tal que $\mathbf{z}(t_1)=\mathbf{z}_1$ se e somente se 
$\mathbf{z}_1 - \mathbf{z}_0 \in \text{Im}(W(t_0,t_1)) = \text{Im}(\int_{t_0}^{t_1}B(s)B^T(s)ds)$
\end{frame}

\begin{frame}
  De fato, se $w\in \mathbb{R}^n $ é tal que $W(t_0,t_1)=\mathbf{z}_1 -\mathbf{z}_2$
  então podemos tomar 
  \begin{gather*}
    u(t) = B^T(t)w \text{ poi neste caso }\\
    \int_{t_0}^{t_1}B(s)u(s)ds = \int_{t_0}^{t_1}B(s)B^T(s)wds = W(t_0,t_1)w.
  \end{gather*}
\end{frame}
\end{document}
