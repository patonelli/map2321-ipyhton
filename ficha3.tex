\documentclass{beamer}
%
% Choose how your presentation looks.
%
% For more themes, color themes and font themes, see:
% http://deic.uab.es/~iblanes/beamer_gallery/index_by_theme.html
%
\mode<presentation>
{
  \usetheme{default}      % or try Darmstadt, Madrid, Warsaw, ...
  \usecolortheme{default} % or try albatross, beaver, crane, ...
  \usefonttheme{default}  % or try serif, structurebold, ...
  \setbeamertemplate{navigation symbols}{}
  \setbeamertemplate{caption}[numbered]
} 

\usepackage[brazil]{babel}
\usepackage[utf8]{inputenc}
\usepackage[T1]{fontenc}

\usepackage{tikz}
\usepackage{siunitx}

\title[]{Exponencial na unha!}
\author{Tópicos em controle}
\institute{IME USP}
\date{5 de setembro de 2020}

\begin{document}



\begin{frame}
  \titlepage
\end{frame}

\begin{frame}{A Matriz}
  Queremos calcular $\exp(tA)$ onde 
  \begin{gather*}
    A= \begin{bmatrix}
      0 & 1 & 0 & 0 \\
      3\omega^2 & 0 & 0 &2\omega \\
      0 & 0 & 0 & 1 \\
      0 & -2\omega & 0 & 0
    \end{bmatrix}
  \end{gather*}

  A minha estratégia será usar a definição 
  $$ \exp(tA) = \sum_{n=0}^\infty \frac{t^n A^n}{n!}$$
  e tentar inferir as somas dos termos $a_{ij}(t)$ de $\exp(tA)$.

 
  
\end{frame}
\begin{frame}{Contas}
  \begin{gather*}
    \begin{array}{|cccc|cccc|} \hline
      & & A^{(1,3,5,7)}& \rightarrow&0 & 1 & 0 & 0 \\
     & & & & 3\omega^2 & 0 & 0 &2\omega \\
     A^{(2,4,6)}& & & &0 & 0 & 0 & 1 \\
     \downarrow& & & &0 & -2\omega & 0 & 0 \\ \hline
     3\omega^2 & 0 & 0 &2\omega &0 & -\omega^2&0 &0 \\
     0 & -\omega^2 & 0 & 0 &-3\omega^4 &0 &0 &-2\omega^3 \\
     0 & -2\omega & 0 & 0 & -6\omega^3&0 &0 &-4\omega^2 \\
     -6\omega^3 & 0 & 0 &-4\omega^2 &0 &2\omega^3 & 0 & 0 \\ \hline
     -3\omega^4 & 0 & 0 &-2\omega^3 &0 & \omega^4&0 &0 \\
     0 & \omega^4 & 0 & 0 &3\omega^6 &0 &0 &2\omega^5 \\
     0 & 2\omega^3 & 0 & 0 & 6\omega^5&0 &0 &4\omega^4 \\
     6\omega^5 & 0 & 0 &4\omega^4 &0 &-2\omega^5 & 0 & 0 \\ \hline
     3\omega^6 & 0 & 0 &2\omega^5 &0 & -\omega^6&0 &0 \\
     0 & -\omega^6 & 0 & 0 &-3\omega^8 &0 &0 &-2\omega^7 \\
     0 & -2\omega^5 & 0 & 0 & -6\omega^7&0 &0 &-4\omega^6 \\
     -6\omega^7 & 0 & 0 &-4\omega^6 &0 &2\omega^7 & 0 & 0 \\ \hline
    \end{array}
  \end{gather*}

\end{frame}

\begin{frame}{Tabela dos coeficientes - Linhas 1 e 2}
  \begin{gather*}
    \begin{array}{c||c|c|c|c|c|c|c|}
      0 & 1 & 2 & 3 & 4 & 5 & 6 & 7 \\ \hline
      a_{11}&0 &3\omega^2 & 0& -3\omega^4& 0 &3\omega^6& 0 \\ 
      a_{12}& 1& 0& -\omega^2&0 & \omega^4&0 & -\omega^6 \\ 
      a_{13}& 0& 0& 0&0 &0 &0 & 0\\ 
      a_{14}& 0&2\omega &0 &-2\omega^3 &0 & 2\omega^5& 0\\ 
      *& -& -&- & -& -& -& - \\ 
      a_{21}& 3\omega^2& 0 &-3\omega^4 & 0& 3\omega^6 &0 & -3\omega^8 \\
      a_{22}& 0&-\omega^2 & 0 &\omega^4 & 0& -\omega^6& 0 \\ 
      a_{23}& 0 &0 &0 &0 &0 &0 &0  \\ 
      a_{24}&2\omega &0 &-2\omega^3 &0 &2\omega^5 &0 &-2\omega^7  \\ \hline
      * & t & t^2/2! & t^3/3!& t^4/4! & t^5/5! &t^6/6! & t^7/7! 
    \end{array}
  \end{gather*}
\end{frame}

\begin{frame}
  \begin{gather*}
    \begin{array}{c||c|c|c|c|c|c|c|}
      0 & 1 & 2 & 3 & 4 & 5 & 6 & 7 \\ \hline
      a_{31}&0 & 0 & -6\omega^3& 0 &6\omega^5 & 0& -6\omega^7 \\ 
      a_{32}& 0& -2\omega& 0& 2\omega^3&0 & -2\omega^5& 0  \\ 
      a_{33}&0 & 0& 0& 0& 0& 0& 0  \\ 
      a_{34}&1 & 0& -4\omega^2&0 &4\omega^4 & 0 & -4\omega^6\\ 
      *& -& -&- & -& -&- & - \\ 
      a_{41}& 0& -6\omega^3& 0 & 6\omega^5&0 &-6\omega^7 & 0 \\
      a_{42}&-2\omega &0 &2\omega^3 &0 & -2\omega^5& 0& 2\omega^7\\ 
      a_{43}&0 &0 & 0& 0& 0& 0& 0 \\ 
      a_{44}& 0&-4\omega^2 &0 &4\omega^4 & 0&-4\omega^6 &0 \\ \hline 
      * & t & t^2/2! & t^3/3!& t^4/4! & t^5/5! &t^6/6! & t^7/7!  
    \end{array}
  \end{gather*}
\end{frame}

\begin{frame}{Cálculo do $a_{11}$}
 Agora 
 $\exp(tA)=I + \sum_{n=1}^\infty \frac{t^n A^n}{n!}$ então para $a_{11}$ temos:
 
 \begin{gather*}
   a_{11} =1 + 3(\frac{(\omega t)^2}{2!} - \frac{(\omega t)^4}{4!} + \frac{(\omega t)^6}{6!}) = \\
   = 1 - 3(\sum_{k=1}^\infty \frac{(i\omega t)^{2k}}{(2k)!})\\
   s_1(t)=\text{e}^{i\omega t} = \sum \frac{(i\omega t)^n}{n!} \\
   s_2(t) =\text{e}^{-i\omega t} = \sum \frac{(-i\omega t)^n}{n!} \\
   \sum_{k=1}^\infty \frac{(i\omega t)^{2k}}{(2k)!} = \frac{s_1(t)+s_2(t)}{2} -1 = \cos(\omega t)-1\\
   a_{11}(t) = 1 -3(\cos(\omega t)-1)= 4-3\cos(\omega t)
\end{gather*}
Os outros coeficientes ficam como exercício!
\end{frame}

\end{document}
