\documentclass[12pt]{article}
\usepackage{amsfonts, amsmath, amssymb}
\usepackage[brazil]{babel}
\usepackage{graphicx}
\usepackage[T1]{fontenc}
\usepackage{ae}

\parindent=0pt

 \addtolength{\textheight}{3.5cm}
 \addtolength{\oddsidemargin}{-1cm}
 \addtolength{\evensidemargin}{-1cm}
 \addtolength{\textwidth}{2cm}
 \addtolength{\topmargin}{-2.0cm}
\newcounter{questao}
\newcommand{\quest}{\stepcounter{questao}{\bf \arabic{questao}.\ }}

\begin{document}
\hrule
 {  \sf  Lista de exerc�cios de MAP2321 \hfill \fbox{2-15}}
\hrule

\vspace{0.5cm}

\thispagestyle{empty}

\quest Verificar a controlabilidade e colocar na forma normal de Kalman os pares $(A,B)$ dados abaixo.

\begin{gather*}
  A=
  \begin{pmatrix}
    -22.2 & -10.6 & -25.2\\
   5.2  &  2.6  &  7.2 \\
  16.6 &   7.8 &  18.6
\end{pmatrix}\text{ e } B=
\begin{pmatrix}
  -0.8\\
  0.8\\
  0.4\\
\end{pmatrix}
\end{gather*}

\quest Sejam $A$ uma matriz $n\times n$ e $B$ uma matriz $n\times m$. Defimos $V\subset \mathbb{R}^n$ como o menor subespa�o de $\mathbb{R}^n$ que cont�m a imagem de $B$ e � invariante por $A$. Mostre que o par $(A,B)$ � control�vel se, e somente se $V=\mathbb{R}^n$. 

\quest Verifique se o seguinte sistema linear � observ�vel:
\begin{eqnarray*}
  \dot{x}_1 &=& 2x_1 -x_2 +x_3 + u_1 \\
  \dot{x}_2&=& x_2 +4x_3 \\
  \dot{x_3} &=& -x_1 +2x_3 +u_2 \\
 & & \\[-1em]
y&=& x_1+x_2
\end{eqnarray*}

\quest Considere a matriz $A$ abaixo. Calcule $\omega(A)$ e, se poss�vel um n�mero positivo $M$ tal que $||\text{e}^{tA}x|| \leq M\text{e}^{5t}||x||$ para todo $x$ e $t>0$.
\begin{gather}
  A=
  \begin{pmatrix}
    2 & -1 \\ 5 & 5
  \end{pmatrix}
\end{gather}

\quest Suponha que $\lambda_1 = -2\mathbf{i}$ esteja no espectro da matriz $A$, ent�o existe um vetor $x_o$ de $\mathbb{R}^n$ tal que $\| \text{e}^{tA}x_0\|$ n�o vai a zero quando $t$ vai a infinito.

\quest Verifique se o polin�mio $p(x) = x^6 +120 x^5 + 650 x^4 + 20 x^3 + 16x^2 +248x + 6 $ � est�vel.

\quest Use a matriz de Hurwitz para dar uma condi��o necess�ria e suficiente para a estabilidade de um polin�mio de grau 4.

\quest $A$ e $B$ s�o as matrizes dadas abaixo. Verifique que $(A,B)$ � control�vel. Ache o polin�mio caracter�stico de $A$ na forma $p(\lambda)= \lambda^3 + a_1\lambda^2 + a_2\lambda + a_3$ e ache a matriz invers�vel $P$ tal que 
\begin{gather}
  P^{-1}AP =
  \begin{pmatrix}
    0&1&0 \\ 0&0&1 \\ -a_3&-a_2&-a_1
  \end{pmatrix}\text{ e } P^{-1}B =
  \begin{pmatrix}
    0 \\ 0 \\ 1
  \end{pmatrix}
\end{gather} 
 onde 
\begin{gather}
A=
\begin{pmatrix}
  2 & 3&  1\\
 0 & 4 & 2 \\
 1 & 0 & 2
\end{pmatrix}\text{ e } B=
\begin{pmatrix}
  2 \\ 3 \\ 1
\end{pmatrix}
\end{gather}

\quest Mostre que se o par $(A,B)$ n�o � completamente estabiliz�vel ent�o o conjunto $\{ \omega(A+BK) : K\in \mathbb{R}^{m\times n}\}$ � limitado inferiormente.

\quest Usando o par $(A,B)$ do exerc�cio anterior, encontre uma matriz $K$ de dimens�o $1\times 3$ talque $\|\text{e}^{(A+BK)}x\| \leq 0.1\|x\|.$

\end{document}

%%% Local Variables: 
%%% mode: latex
%%% TeX-master: t
%%% End: 
